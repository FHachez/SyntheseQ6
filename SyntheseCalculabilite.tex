\documentclass[11pt,a4paper]{article}
\usepackage[utf8]{inputenc}
\usepackage[french]{babel}
\usepackage[T1]{fontenc}
\usepackage{amsmath}
\usepackage{amsfonts}
\usepackage{amssymb}
\usepackage{graphicx}
\usepackage{pdfpages}
\usepackage{palatino}
\usepackage{listingsutf8} 
\usepackage{xcolor}
\usepackage{enumitem}
\usepackage{tabularx}
\usepackage{tikz}
\usepackage{placeins}

\usepackage[bottom]{footmisc}
\usepackage{desclist}
\usepackage{tocloft}
\setlength\cftparskip{4pt}

\setlist[itemize]{topsep=3pt,after=\vspace{.5\baselineskip}}
\usepackage[left=3cm,right=3cm,top=3cm,bottom=3cm]{geometry}
\setlength{\parskip}{2mm}

\definecolor{mygreen}{rgb}{0,0.6,0}
\definecolor{mygray}{rgb}{0.5,0.5,0.5}
\definecolor{mymauve}{rgb}{0.58,0,0.82}

\renewcommand{\stcomp}[1]{\overline{#1}} 

% Modification de 'listing'
\lstset{
  language=Java,						% choose the language of the code
  numbers=left,						% where to put the line-numbers
  stepnumber=1,						% the step between two line-numbers.        
  numbersep=5pt,						% how far the line-numbers are from the code
  backgroundcolor=\color{white},		% choose the background color. You must add \usepackage{color}
  showspaces=false,					% show spaces adding particular underscores
  showstringspaces=false,			% underline spaces within strings
  showtabs=false,					% show tabs within strings adding particular underscores
  tabsize=2,							% sets default tabsize to 2 spaces
  captionpos=b,						% sets the caption-position to bottom
  breaklines=true,					% sets automatic line breaking
  breakatwhitespace=true,			% sets if automatic breaks should only happen at whitespace
  title=\lstname,					% show the filename of files included with \lstinputlisting;
  keywordstyle=\color{blue},			% keyword style
  numberstyle=\tiny\color{mygray},	% the style that is used for the line-numbers
  rulecolor=\color{black},			% the frame-color may be changed on line-breaks within not-black text
  basicstyle=\small\ttfamily,		% text style
  basewidth=0.51em,					% text height
  showstringspaces=false,			% show space
  frame=single,						% type of border
  commentstyle=\color{mygreen},		% comment style
  stringstyle=\color{mymauve},		% string literal style
  tabsize=2,							% sets default tabsize to 2 spaces
  belowskip=-2mm,
}


\def\blurb{\textsc{Université catholique de Louvain\\
  École polytechnique de Louvain}}
\def\clap#1{\hbox to 0pt{\hss #1\hss}}%
\def\ligne#1{%
  \hbox to \hsize{%
    \vbox{\centering #1}}}%
\def\haut#1#2#3{%
  \hbox to \hsize{%
    \rlap{\vtop{\raggedright #1}}%
    \hss
    \clap{\vbox{\vfill\centering #2\vfill}}%
    \hss
    \llap{\vtop{\raggedleft #3}}}}%
\begin{document}

\begin{titlepage}
\thispagestyle{empty}\vbox to 1\vsize{%
  \vss
  \vbox to 1\vsize{%
    \haut{\raisebox{-8mm}{\includegraphics[width=2cm]{logo_ucl.pdf}}}{\blurb}{\raisebox{-3mm}{\includegraphics[scale=0.35]{logo_epl.jpg}}}
    \vfill
    \ligne{\Huge \textbf{\textsc{Synthèse calculabilité}}}
    \vspace{5mm}
    \ligne{\large{-- juin 2014 --}}
    \vfill
    \ligne{%
      \begin{tabular}{c}
        \textsc{Travail du groupe :}
      \end{tabular}}
    \vspace{5mm}
    \ligne{%
         \textsc{Hachez} Floran   7372-11-00 
      }
    }%
  \vss
  }
\end{titlepage}

\tableofcontents
\newpage

\section{Introduction}
\label{sec:introduction}

\paragraph{}
\label{par:}
La calculabilité c'est l'étude des limites de l'informatique. Il faut bien 
faire attention à faire la différence entre les limites théoriques et les limites
pratiques. Pour calculabilité, on s'occupe des limites théoriques.
Alors que pour la complexité on s'occupe des limites pratiques, détermine la frontière
entre faisable en pratique et infaisable en pratique.
La question principale de la calculabilité est quels sont les problèmes qui peuvent
être résolus par un programme et lesquels ne peuvent pas.

\paragraph{} Le but est de tracer des frontières entre les programmes calculables,
non calculables et non calculables en pratique.

\paragraph{}
\label{par:}
L'intérêt pratique est qu'on peut donc savoir quand ça ne sert à rien 
de résoudre un problème. De plus, on est conscient de a complexité intrinsèque d'un
problème.
% paragraph  (end)

\subsection{Notion de problème}
\label{subsec:notion_de_probl_me}

\paragraph{}
\label{par:}
Premièrement, on doit parler la notion de problème.
Attention, il ne faut pas confondre un problème avec un programme.
Les caractéristiques sont:

\begin{itemize}
	\item un problème est générique : il s'applique à un ensemble de données.
	\item pour chaque donnée particulière, il existe une réponse.
\end{itemize}
On représente un problème dans le cours par une fonction. Donc dans le cours,
la description d'un problème est équivalente à la description delà fonction.
% paragraph  (end)
% subsection notion_de_probl_me (end)

\subsection{Notion de programme}
\label{ssub:notion_de_programme}

Un programme est une "procédure effective", c'est-à-dire exécutable par une machine.
Il existe plein de formalisme permettant la description de "procédure effective".

% subsection notion_de_programme (end)

\subsection{Résultats principaux}
\label{sub:r_sultat_principaux}

\begin{itemize}
	\item Équivalence des langages de programmation (Complet).
	\item Problème non calculable : Il existe des problèmes qui ne peuvent 
		être résolus par un programme. Ex: détection de virus, équivalence
		de programme,...
	\item Problème intrinsèquement complexe. (Voir complexité) Les problèmes
		qui ont une complexité supérieure ou égale à l'exponentielle. Dans
		ce cas même l'amélioration des ordinateurs n'influence presque pas
		la taille de l'entrée possible.
\end{itemize}

% subsection r_sultat_principaux (end)

\subsection{Détection de Virus}
\label{sub:d_tection_de_virus}
On veut déterminer si un programme P avec une entrée D est nuisible.

Spécification du programme détecteur(P,D):\\
Préconditions : un programme P et une donnée D\\
Postconditions : "Mauvais" si P(D) est nuisible,
		"Bon" sinon

\paragraph{}On va créer un programme drôle(P) et essayer de détecter s’ il est nuisible.

drôle(P) \\
if( détecteur(P,P)="Mauvais" then stop \\
else infecter un autre programme en y insérant P

\paragraph{}
Testons drôle(drôle). \\
drôle(drôle) \\
if( détecteur(drôle, drôle)="Mauvais" then stop \\
else infecter un autre programme en y insérant drôle \\

Si drôle(drôle) est nuisible alors le programme s'arrête or il n'est pas nuisible
puisqu'il n'a infecté aucun programme.\\
Si par contre il n'est pas nuisible alors il va un infecter un autre programme.\\
On a donc une contradiction ce qui implique que le programme drôle ne peut exister,
le programme détecteur non plus.
% paragraph  (end)
% subsection d_tection_de_virus (end)

% section introduction (end)Introduction

\section{Concepts}
\label{sec:concepts}

% Dans cette partie il y a pas moyen de synthétiser beaucoup.

\subsection{Ensembles, langages, relations et fonctions}
\label{sub:ensembles_langages_relations_et_fonctions}

\subsubsection{Ensembles}
\label{ssub:ensembles}
Un ensemble est une collection d'objets, sans répétition, appelés les éléments
de l'ensemble.\\

Notation : 
\begin{itemize}
	\item Ensemble fini : { 0, 1, 2}
	\item Ensemble infini : { 0, 1, 2, ...}
	\item Produit cartésien : A x B
	\item Complément : $\stcomp{A}$
\end{itemize}

% subsubsection ensembles (end)

\subsubsection{Langages}
\label{ssub:Langages}
Notation : 
\begin{itemize}
	\item une chaine de caractère ou un mot : séquence FINIE de symboles. 
		abceced, 010101101
	\item chaine de caractères vide : $\epsilon$
	\item un alphabet $\sum$ est un ensemble de symboles. $\sum = {1, 2}$
	\item un langage est un ensemble de mots constitués de symboles d'un alphabet
		donné.
	\item ensemble de tous les mots possible avec $\sum$ : $\sum ^*$
\end{itemize}

% subsubsection Langages (end)

\subsubsection{Relations}
\label{ssub:relations}
Soient A, B des ensembles.
\begin{itemize}
	\item Une relation R sur A, B est un sous-ensemble de A x B. C'est-à-dire
		un ensemble de paires (a,b) avec $a\in A$ $b\in B$.
	\item On peut définir une relation par sa table
\end{itemize}

% subsubsection relations (end)



% subsection ensembles_langages_relations_et_fonctions (end)


% section concepts (end)

\end{document}
